\documentclass[../main.tex]{subfiles}

\begin{document}

%%%%%%%%%%%%%%%%%%%%%%%%%%%%%%%%%%%%%%%%%%%%%%%%%%%%%%%%%%%%%%%%%%%%%%%%%%%%%%%%%%%%%

\section{''Matrix''-feltet}

Betragt en mængde af felter $\phi_{ij}$, hvor $i,j = 1,\ldots,N$, og antag at $\phi_{ij} = \phi_{ji}$. Antag yderligere at felterne er reelle, altså at $\phi_{ij} \in \R,\,\: \forall i,j=1,\ldots,N$. Lagrangedensiteten for dette system er
\begin{align} \label{eq:Opg2_Lagrangian}
    \L &= \inv{2} \partial_\mu \phi_{ij} \partial^\mu \phi_{ji} - \inv{2} m^2 \phi_{ij} \phi_{ji} \: ,
\end{align}
hvor vi har implicitte summer over felterne således at
\begin{align} \label{eq:Opg2_ImplicitSums}
    \phi_{ij} \phi_{jk} &= \sum_{j=1}^N \phi_{ij}\phi_{jk} \quad \text{og} \quad
    \phi_{ij} \phi_{ji} = \sum_{i=1}^N \sum_{j=1}^N \phi_{ij}\phi_{ji} \: .
\end{align}

Bemærk hvordan $\phi_{ij} \phi_{ji}$ svarer til at betragte feltet som værende en matrix, multiplicere matricen med sig selv og sidst tage trace af den resulterende matrix.


%%%%%%%%%%%%%%%%%%%%%%%%%%%%%%%%%%%%%%%%%%%%%%%%%%%%%%%%%%%%%%%%%%%%%%%%%%%%%%%%

\paragraph*{\textbf{1)}}

Hvor mange frihedsgrader har feltet $\phi_{ij}$?


%%%%%%%%%%%%%%%%%%%%%%%%%

\paragraph*{\textbf{2)}}

Vis at bevægelsesligningerne er
\begin{align} \label{eq:Opg2_Q2_EquationOfMotion}
    0 &= \left(\partial_\mu \partial^\mu + m^2\right)\phi_{ij} \: , \quad \forall i,j=1,\ldots,N \: .
\end{align}


%%%%%%%%%%%%%%%%%%%%%%%%%

\paragraph*{\textbf{3)}}

Vis at løsningerne er planbølger på formen $\phi_{ij} = M_{ij}\exp(\pm ipx)$, hvor $px = p^\mu x_\mu = p_\mu x^\mu$ med $p^\mu = (E,\, \vv{p})$ og $M_{ij}$ er uafhængig af $x^\mu$. Find udtrykket for E.


%%%%%%%%%%%%%%%%%%%%%%%%%

\paragraph*{\textbf{4)}}

Definer det konjugerede felt, $\pi_{ij}$, som den naturlige generalisering af enkeltkomponent skalarfelter
\begin{align} \label{eq:Opg2_Q4_ConjugatedMomentaField}
    \pi_{ij} &= \pdif{\L}{(\partial_0 \phi_{ij})} \: .
\end{align}
Vis at $\pi_{ij} = \partial^0 \phi_{ij}$.


%%%%%%%%%%%%%%%%%%%%%%%%%

\paragraph*{\textbf{5)}}

Vis at Hamiltonfunktionen er
\begin{align} \label{eq:Opg2_Q5_Hamiltonian}
    H &= \int \dd^3\vv{x} \left[ \inv{2} \pi_{ij} \pi_{ji} + \inv{2} \Grad{\phi_{ij}} \cdot \Grad{\phi_{ji}} + \inv{2} m^2 \phi_{ij} \phi_{ji} \right] \: .
\end{align}


%%%%%%%%%%%%%%%%%%%%%%%%%

\paragraph*{\textbf{6)}}

Argumentér for at en fornuftig feltudvidelse (eng: field expansion) for det kvantemekaniske ''matrix''-felt er
\begin{align} \label{eq:Opg2_Q6_ResultingField}
    \phi_{ij} &= \int \frac{\dd^3\vv{p}}{(2\pi)^3} \invsqrt{2\omega_{\vv{p}}} \left(a_{\vv{p}ij} \bexp{-ipx} + a_{\vv{p}ij}\dagger \bexp{ipx} \right) \: ,
\end{align}
hvor $\omega_{\vv{p}} = \sqrt{\vv{p}^2 + m^2}$.


%%%%%%%%%%%%%%%%%%%%%%%%%

\paragraph*{\textbf{7)}}

Vi introducerer nu et indre produkt, kaldet Klein-Gordon-indreproduktet, mellem to bølgefunktioner på en lidt besynderlig måde. Lad $f$ og $g$ være funktioner af tid og rum og definer
\begin{align}
\begin{split} \label{eq:Opg2_Q7_KGInnerProduct}
    (f,g) &= i \int \dd^3\vv{x} f^*(\vv{x},\, t) \overset{\leftrightarrow}{\partial_0} g(\vv{x},\, t) \\
        &= i \int \dd^3\vv{x} \left[ f^*(\vv{x},\, t) \pdif{g(\vv{x},\, t)}{t} - \pdif{f^*(\vv{x},\, t)}{t} g(\vv{x},\, t) \right] \: ,
\end{split}
\end{align}
hvor $\overset{\leftrightarrow}{\partial_0}$ short-hand notation for operatoren, som tidsdifferentierer til først til højre og så til venstre med en minus imellem. Bemærk at dette indre produkt ikke er positivt bestemt (eng: positive definite) og derfor ikke er et matematisk indre produkt, hvis man skal være stringent. Men det er en dejlig og nyttig konstruktion, hvilket vi nu skal vise.

Betragt en planbølge på formen $u_{\vv{p}}(x) = \pexp{-ipx}/\sqrt{2\omega_{\vv{p}}}$, hvor $x$ og $p$ er firvektorer som sædvanligt. Vis de følgende relationer
\begin{subequations} \label{eq:Opg2_Q7_KGInnerProductsWithU}
\begin{align}
    (u_{\vv{p}}, u_{\vv{p}'}) &= (2\pi)^3 \delta^3(\vv{p} - \vv{p}') \: , \\
    (u_{\vv{p}}^*, u_{\vv{p}'}^*) &= - (2\pi)^3 \delta^3(\vv{p} - \vv{p}') \: , \\
    (u_{\vv{p}}^*, u_{\vv{p}'}) &= 0 \: , \quad \text{og} \\
    (u_{\vv{p}}, u_{\vv{p}'}^*) &= 0 \: .
\end{align}
\end{subequations}


%%%%%%%%%%%%%%%%%%%%%%%%%

\paragraph*{\textbf{8)}}

Benyt nu det indre produkt med feltet fra \cref{eq:Opg2_Q6_ResultingField} og vis at
\begin{subequations} \label{eq:Opg2_Q8_StepOperatorsUsingKGInnerProduct}
\begin{align}
    a_{\vv{p}ij} &= (u_{\vv{p}},\phi_{ij}) \: , \quad \text{og} \\
    a_{\vv{p}ij}\dagger &= -(u_{\vv{p}}^*,\phi_{ij}) \: .
\end{align}
\end{subequations}


%%%%%%%%%%%%%%%%%%%%%%%%%

\paragraph*{\textbf{9)}}

Kommutatorrelationerne for matrix-felterne er givet ved
\begin{align}
    \commutator{\phi_{ij}(\vv{x},\, t)}{\pi_{kl}(\vv{x}',\, t)} &= i\delta^3(\vv{x} - \vv{x}')\delta_{ik}\delta{jl} \: ,
\end{align}
mens alle andre kombinationer af $\phi_{ij}$ og $\pi_{kl}$ forsvinder. Vis at dette betyder, at kreations- og annihilationsoperatorerne, $a_{\vv{p}ij}\dagger$ og $a_{\vv{p}ij}$, har bosniske kommutationsrelationer.


%%%%%%%%%%%%%%%%%%%%%%%%%

\paragraph*{\textbf{10)}}

Vis at Hamiltonfunktionen for det kvantiserede felt kan skrives som
\begin{align}
    H &= \sum_{i,j=1}^N \int \frac{\dd^3\vv{p}}{(2\pi)^3} \left[ \omega_{\vv{p}} a_{\vv{p}ij}\dagger a_{\vv{p}ij} \right] \: .
\end{align}


%%%%%%%%%%%%%%%%%%%%%%%%%

\paragraph*{\textbf{11)}}

Vis at feltoperatoren overholder Heisenbergs bevægelsesligning
\begin{align}
    i\pdif{\phi_{ij}}{t} &= \commutator{\phi_{ij}}{H} \: .
\end{align}

Argumentér for at dette betyder, at feltoperatorerne fra \cref{eq:Opg2_Q6_ResultingField} er de rigtige feltoperatorer at benytte i vekselvirkningsbilledet (eng: interaction picture).


%%%%%%%%%%%%%%%%%%%%%%%%%

\paragraph*{\textbf{12)}}

Vi vil nu kigge på symmetrier for vores matrix-felter. Betragt gruppen af ortogonale $N \times N$ matricer, hvilken kaldes $O(N)$ og defineres som mængden
\begin{align}
    O(N) &= \left\{ R \:\vert\: R^\intercal R = \id \right\} \: ,
\end{align}
hvor $R$ er en reel $N \times N$ matrix og $R^\intercal$ betegner den transponerede matrix af $R$.

Vælg $R \in O(N)$ og lad komponenterne betegnes $R_{ij}$. Vis at Lagrangedensiteten i \cref{eq:Opg2_Lagrangian} er invariant under transformationen
\begin{align}
    \phi_{ij}' &= R_{ik}R_{jl}\phi_{kl} \: .
\end{align}
Dette kaldes $O(N)^2$ symmetri.


%%%%%%%%%%%%%%%%%%%%%%%%%

\paragraph*{\textbf{13)}}

Find et vekselvirkningsled (eng: interaction term) som kan lægges til Lagrangedensiteten således, at denne stadig har $O(N)^2$ symmetri. Find hernæst et vekselvirkningsled, som eksplicit \emph{bryder} $O(N)^2$ symmetrien for Lagrangedensiteten.


%%%%%%%%%%%%%%%%%%%%%%%%%

\paragraph*{\textbf{14)}}

Betragt et element $R$ i $O(N)$ meget tæt på identitetsmatricen,  således at man kan skrive elementet $R = \id + A$, hvor $A$ er en $N \times N$ matrix med elementer $\abs{A_{ij}} < 1$. Vis at $R \in O(N)$ medfører at $A^\intercal = -A$, altså at $A$ er antisymmetrisk. Mængden af sådanne matricer $A$ kaldes \emph{generatoren} af $O(N)$. Hvor mange lineært uafhængige matrice findes med denne egenskab for dimensionen $N$? Dette kaldes dimensionen af $O(N)$.


%%%%%%%%%%%%%%%%%%%%%%%%%

\paragraph*{\textbf{15)}}

Find den bevarede Noetherstrøm (eng: Noether current) som associeres med $O(N)^2$ symmetrien.


%%%%%%%%%%%%%%%%%%%%%%%%%

\paragraph*{\textbf{16)}}

Hvad ville ændre sig, hvis feltet $\phi_{ij}$ var komplekst i stedet for reelt? Beskriv blot de overordnede forskellige; du skal \emph{ikke} beregne hele opgaven igen med et komplekst felt.


%%%%%%%%%%%%%%%%%%%%%%%%%%%%%%%%%%%%%%%%%%%%%%%%%%%%%%%%%%%%%%%%%%%%%%%%%%%%%%%%

\subsection{Besvarelse}

%%%%%%%%%%%%%%%%%%%%%%%%%

\paragraph[1) Frihedsgrader for feltet $\phi_{ij}$]{\textbf{1)}}




%%%%%%%%%%%%%%%%%%%%%%%%%

\paragraph[2) Bevægelsesligningerne for $\psi_{ij}$]{\textbf{2)}}

Vi betragter først Lagrangedensiteten \cref{eq:Opg2_Lagrangian} og konstaterer, at idet $\phi_{ij} = \phi_{ji}$ samt at vi kan benytte metrikken til at ændre kovariante firvektorer til kontravariante og omvendt, så kan Lagrangedensiteten skrives som
\begin{align}
    \L &= \inv{2} \partial_\mu \phi_{ij} \partial^\mu \phi_{ji} - \inv{2} m^2 \phi_{ij} \phi_{ji}
        = \inv{2} g^{\mu\nu} \partial_\mu \phi_{ij} \partial_\nu \phi_{ij} - \inv{2} m^2 \phi_{ij} \phi_{ij} \: .
\end{align}

For at finde bevægelsesligningerne benytter vi Euler-Lagrangeligningen
\begin{align} \label{eq:Opg2_A2_EulerLagrangeEquation}
    0 &= \partial_\mu \pdif{\L}{(\partial_\mu \phi_{ij})} - \pdif{\L}{\phi_{ij}}
\end{align}
for $i,j = 1,\ldots,N$, hvormed vi altså får $i \cdot j$ bevægelsesligninger.

Først beregner vi den afledede mht. feltet selv
\begin{align} \label{eq:Opg2_A2_EulerLagrangeEquation_PartPotential}
\begin{split}
    \pdif{\L}{\phi_{ij}} &= \pdif{}{\phi_{ij}} \left( \inv{2} g^{\rho\nu} \partial_\rho \phi_{ij} \partial_\nu \phi_{ij} - \inv{2} m^2 \phi_{ij} \phi_{ij} \right) \\
        &= \pdif{}{\phi_{ij}} \left( - \inv{2} m^2 \phi_{ij}^2 \right) \\
        &= - \inv{2} m^2 2 \phi_{ij} \\
        &= - m^2 \phi_{ij} \: ,
\end{split}
\end{align}
og hernæst den afledede af Lagrangedensiteten mht. det afledede feltet
\begin{align} \label{eq:Opg2_A2_EulerLagrangeEquation_PartKinetic}
\begin{split}
    \pdif{\L}{(\partial_\mu \phi_{ij})} &= \pdif{}{(\partial_\mu \phi_{ij})} \left( \inv{2} g^{\rho\nu} \partial_\rho \phi_{ij} \partial_\nu \phi_{ij} - \inv{2} m^2 \phi_{ij} \phi_{ij} \right) \\
        &= \pdif{}{(\partial_\mu \phi_{ij})} \left( \inv{2} g^{\rho\nu} \partial_\rho \phi_{ij} \partial_\nu \phi_{ij} \right) \\
        &= \inv{2} g^{\rho\nu} \left( \pdif{}{(\partial_\mu \phi_{ij})} \big( \partial_\rho \phi_{ij} \big) \partial_\nu \phi_{ij} + \partial_\rho \phi_{ij} \pdif{}{(\partial_\mu \phi_{ij})} \big( \partial_\nu \phi_{ij} \big) \right) \\
        &= \inv{2} g^{\rho\nu} \left( \partial_\nu \phi_{ij} + \partial_\rho \phi_{ij} \right) \\
        &= \inv{2} \left( \partial^\rho \phi_{ij} + \partial^\nu \phi_{ij} \right) \\
        &= \inv{2} \left( \partial^{\cancelto{\nu}{\rho}} \phi_{ij} + \partial^\nu \phi_{ij} \right) \\
        &= \partial^\nu \phi_{ij} \: .
\end{split}
\end{align}

Indsætter vi nu de beregnede afledede, \cref{eq:Opg2_A2_EulerLagrangeEquation_PartPotential,eq:Opg2_A2_EulerLagrangeEquation_PartKinetic}, i Euler-Lagrangeligningen, \cref{eq:Opg2_A2_EulerLagrangeEquation}, fås bevægelsesligningen for feltet $\phi_{ij}$ til
\begin{align}
\begin{split}
    0 &= \partial_\mu \big( \partial^\mu \phi_{ij} \big) - \big( -m^2 \phi_{ij} \big) \\
        &= \big( \partial_\mu \partial^\mu + m^2 \big) \phi_{ij} \: .
\end{split}
\end{align}

Altså er bevægelsesligningerne
\begin{align}
    0 &= \left(\partial_\mu \partial^\mu + m^2\right)\phi_{ij} \: , \quad \forall i,j=1,\ldots,N \: .
\end{align}
som givet af \cref{eq:Opg2_Q2_EquationOfMotion}.


%%%%%%%%%%%%%%%%%%%%%%%%%

\paragraph[3) Løsningerne for felterne er planbølger]{\textbf{3)}}

For at vise, at løsningerne til \cref{eq:Opg2_Q2_EquationOfMotion} er planbølger på formen $\phi_{ij} = M_{ij}\exp(\pm ipx)$, hvor $px = p^\mu x_\mu = p_\mu x^\mu$ med $p^\mu = (p^0,\, p^a) = (E,\, \vv{p})$ og $M_{ij}$ er uafhængig af $x^\mu$, indsættes disse planbølger i bevægelsesligningen \cref{eq:Opg2_Q2_EquationOfMotion}
\begin{align}
\begin{split}
    \left(\partial_\mu \partial^\mu + m^2\right)\phi_{ij} &= \left(\partial_\mu \partial^\mu + m^2\right) M_{ij}\pexp{\pm ipx} \\
        &= M_{ij} \left(\partial_\mu \partial^\mu + m^2\right) \pexp{\pm ipx} \\
        &= M_{ij} \Big(\partial_\mu \big[ \pm i \pexp{\pm ipx} p^\mu \big] + \pexp{\pm ipx} m^2\Big) \\
        &= M_{ij} \Big(\pm i \big[ \partial_\mu \pexp{\pm ipx} \big] p^\mu + \pexp{\pm ipx} m^2\big) \\
        &= M_{ij} \Big([\pm i]^2 \pexp{\pm ipx} p_\mu p^\mu + \pexp{\pm ipx} m^2\Big) \\
        &= M_{ij} \pexp{\pm ipx} \Big(- p_\mu p^\mu + m^2\Big) \\
        &= \phi_{ij} \Big(- p_0 p^0 - p_a p^a + m^2\Big) \\
        &= \phi_{ij} \Big(- EE - \big[-\vv{p}\big] \cdot \vv{p} + m^2\Big) \\
        &= \phi_{ij} \Big(- E^2 + \vv{p}^2 + m^2\Big) \\
        &= \phi_{ij} \Big(- E^2 + E^2 \Big) \\
        &= 0 \: ,
\end{split}
\end{align}
hvor den relativistiske energi er $E = \sqrt{\vv{p}^2 + m^2}$. Altså er planbølger på formen $\phi_{ij} = M_{ij}\exp(\pm ipx)$ løsninger til bevægelseligningen fra \cref{eq:Opg2_Q2_EquationOfMotion}.


%%%%%%%%%%%%%%%%%%%%%%%%%

\paragraph[4) Konjugeret felt er $\pi_{ij}  = \partial^0 \psi_{ij}$]{\textbf{4)}}

Vi er blevet givet, at det konjugerede felt er
\begin{align} \label{eq:Opg2_A4_ConjugatedMomentaField_RepeatedInAnswer}
    \pi_{ij} &= \pdif{\L}{(\partial_0 \phi_{ij})} \: ,
\end{align}
samt at $\phi_{ij} = \phi_{ji}$. Vi indsætter derfor blot Lagrangedensiteten fra \cref{eq:Opg2_Lagrangian} i ligningen for det konjugerede felt (\cref{eq:Opg2_Q4_ConjugatedMomentaField,eq:Opg2_A4_ConjugatedMomentaField_RepeatedInAnswer}) (step a), benytter at indekserne på feltet $\phi_{ij}$ kan byttes rundt uden omkostning (step b), samt benytter produktreglen til at differentiere produktfunktionen (step c).
\begin{align}
\begin{split}
    \pi_{ij} &= \pdif{\L}{(\partial_0 \phi_{ij})} \\
        &\xleq{(a)} \pdif{}{(\partial_0 \phi_{ij})} \left( \inv{2} \partial_\mu \phi_{ij} \partial^\mu \phi_{ji} - \inv{2} m^2 \phi_{ij} \phi_{ji} \right) \\
        &= \pdif{}{(\partial_0 \phi_{ij})} \left( \inv{2} \partial_\mu \phi_{ij} \partial^\mu \phi_{ji} \right) \\
        &\xleq{(b)} \pdif{}{(\partial_0 \phi_{ij})} \left( \inv{2} \partial_\mu \phi_{ij} \partial^\mu \phi_{ij} \right) \\
        &= \pdif{}{(\partial_0 \phi_{ij})} \left( \inv{2} \partial_0 \phi_{ij} \partial^0 \phi_{ij} \right) \\
        &= \pdif{}{(\partial_0 \phi_{ij})} \left( \inv{2} g^{00} \partial_0 \phi_{ij} \partial_0 \phi_{ij} \right) \\
        &= \inv{2} g^{00} \pdif{}{(\partial_0 \phi_{ij})} \left( \partial_0 \phi_{ij} \partial_0 \phi_{ij} \right) \\
        &\xleq{(c)} \inv{2} g^{00} \left( \pdif{}{(\partial_0 \phi_{ij})} \left[\partial_0 \phi_{ij}\right] \partial_0 \phi_{ij} + \partial_0 \phi_{ij} \pdif{}{(\partial_0 \phi_{ij})} \left[\partial_0 \phi_{ij}\right] \right) \\
        &= \inv{2} g^{00} \left( \partial_0 \phi_{ij} + \partial_0 \phi_{ij} \right) \\
        &= g^{00} \partial_0 \phi_{ij} \\
        &= \partial^0 \phi_{ij} \: .
\end{split}
\end{align}
Det er dermed vist, at det konjugerede felt er den kontravariante tidsafledede af feltet, $\pi_{ij} = \partial^0 \phi_{ij}$.


%%%%%%%%%%%%%%%%%%%%%%%%%

\paragraph[5) Hamiltonfunktion med felter]{\textbf{5)}}

Hamiltonfunktionen er givet som
\begin{align} \label{eq:Opg2_A5_HamiltonianDefinition}
    H &= \int \dd^3\vv{x} \H \: ,
\end{align}
hvor $\H$ er Hamiltondensiteten defineret som
\begin{align} \label{eq:Opg2_A5_HamiltonianDensityDefinition}
    \H &= \sum_{ij} \pi_{ij} \left( \partial_0 \phi_{ij} \right) - \L \: .
\end{align}

I kurset benyttes metrikken $g_{\mu\nu} = \mathrm{diag}(1,\, -\id)$, altså at de rummelige koordinater i de kovariante firvektorer får et negativt fortegn, mens tidskoordinatet forbliver det samme. For differentialoperatorer er det dog den kontravariante firvektors rummelige koordinater, som får det negative fortegn, da
\begin{align}
    \partial^\mu &= \pdif{}{x_\mu} = \left( \partial_0,\, -\Grad{} \right) \: .
\end{align}
Af denne grund kan vi også skrive $\pi_{ij}$ som
\begin{align}
    \pi_{ij} &= \partial^0 \phi_{ij}
        = g^{00} \partial_0 \phi_{ij}
        = 1 \partial_0 \phi_{ij}
        \partial_0 \phi_{ij} \: .
\end{align}
Samtidig har vi, at $\phi_{ij} = \phi_{ji}$, hvorfor
\begin{align}
    \pi_{ij} &= \partial^0 \phi_{ij}
        = \partial^0 \phi_{ji}
        = \pi_{ji} \: .
\end{align}
Sammensættes dette kan vi skrive $\pi_{ji} = \partial_0 \phi_{ji}$.

Nu beregnes Hamiltondensiteten ved \cref{eq:Opg2_A5_HamiltonianDensityDefinition}, hvor der ligesom for $\phi_{ij} \phi_{ji}$ (\cref{eq:Opg2_ImplicitSums}) er en implicit sum for $\pi_{ij} \pi_{ji}$, og det romerske bogstav $a$ er brugt for implicit summering over de rummelige koordinater fra firvektoren.
\begin{align}
\begin{split}
    \H &= \sum_{ij} \pi_{ij} \left( \partial_0 \phi_{ij} \right) - \L \\
        &= \pi_{ij} \pi_{ji} - \inv{2} \partial_\mu \phi_{ij} \partial^\mu \phi_{ji} + \inv{2} m^2 \phi_{ij} \phi_{ji} \\
        &= \pi_{ij} \pi_{ji} - \inv{2} \partial_0 \phi_{ij} \partial^0 \phi_{ji} - \inv{2} \partial_a \phi_{ij} \partial^a \phi_{ji} + \inv{2} m^2 \phi_{ij} \phi_{ji} \\
        &= \pi_{ij} \pi_{ji} - \inv{2} \pi_{ij} \pi_{ji} - \inv{2} \Grad{\phi_{ij}} \cdot \left(-\Grad{\phi_{ji}}\right) + \inv{2} m^2 \phi_{ij} \phi_{ji} \\
        &= \inv{2} \pi_{ij} \pi_{ji} + \inv{2} \Grad{\phi_{ij}} \cdot \Grad{\phi_{ji}} + \inv{2} m^2 \phi_{ij} \phi_{ji} \: .
\end{split}
\end{align}

Indsættes denne Hamiltondensitet i \cref{eq:Opg2_A5_HamiltonianDefinition} fås Hamiltonfunktionen
\begin{align}
    H &= \int \dd^3\vv{x} \left[ \inv{2} \pi_{ij} \pi_{ji} + \inv{2} \Grad{\phi_{ij}} \cdot \Grad{\phi_{ji}} + \inv{2} m^2 \phi_{ij} \phi_{ji} \right] \: ,
\end{align}
hvilken var den, som skulle vises (\cref{eq:Opg2_Q5_Hamiltonian}).


%%%%%%%%%%%%%%%%%%%%%%%%%

\paragraph[6) Feltudvidelse for ''matrix''-feltet]{\textbf{6)}}




%%%%%%%%%%%%%%%%%%%%%%%%%

\paragraph[7) Vis relationer med Klein-Gordon-indreproduktet]{\textbf{7)}}

Vi betragter funktionen
\begin{align}
    u_{\vv{p}}(x) &= \frac{\pexp{-ipx}}{\sqrt{2\omega_{\vv{p}}}} \: ,
\end{align}
hvor $x$ og $p$ er firvektorer, $x^\mu = (t,\, \vv{x})$ og $p^\mu = (\omega_{\vv{p}},\, \vv{p})$, og $px = p_\mu x^\mu = p^\mu x_\mu$.

Vi skal nu regne de indre produkter i \cref{eq:Opg2_Q7_KGInnerProductsWithU}, til hvilket vi benytter os af definitionen af Klein-Gordonindreproduktet i \cref{eq:Opg2_Q7_KGInnerProduct}.

For ikke at gentage beregningerne for mange gange betegnes
\begin{align}
    u_{\vv{p}} = u_{\vv{p}}^- \quad \text{og} \quad u_{\vv{p}}^* = u_{\vv{p}}^+
\end{align}
i det følgende. Derved får vi også, at
\begin{align}
    (u_{\vv{p}}^+)^* = (u_{\vv{p}}^*)^* = u_{\vv{p}} = u_{\vv{p}}^- \: .
\end{align}

Først beregnes $(u_{\vv{p}},u_{\vv{p}'})$ og $(u_{\vv{p}}^*,u_{\vv{p}'}^*)$:
\begin{align}
\begin{split}
    (u_{\vv{p}}^\pm,u_{\vv{p}}^\pm) &= i \int \dd^3\vv{x} \left[ u_{\vv{p}}^\mp \pdif{u_{\vv{p}'}^\pm}{t} - \pdif{u_{\vv{p}}^\mp}{t} u_{\vv{p}'}^\pm \right] \\
        &= i \int \dd^3\vv{x} \left[ \frac{\pexp{\mp ipx}}{\sqrt{2\omega_{\vv{p}}}} \pdif{}{t} \left( \frac{\pexp{\pm ip'x}}{\sqrt{2\omega_{\vv{p}'}}} \right) - \pdif{}{t} \left( \frac{\pexp{\mp ipx}}{\sqrt{2\omega_{\vv{p}}}} \right) \frac{\pexp{\pm ip'x}}{\sqrt{2\omega_{\vv{p}'}}} \right] \\
        &= i \int \dd^3\vv{x} \inv{2\sqrt{\omega_{\vv{p}} \omega_{\vv{p}'}}} \Big[ \pexp{\mp ipx} \big\{\pm i \omega_{\vv{p}'} \pexp{\pm ip'x} \big\} \\
            &\qquad\qquad\qquad\qquad\qquad - \big\{ \mp i \omega_{\vv{p}} \pexp{\mp ipx} \big\} \pexp{\pm ip'x} \Big] \\
        &= i \int \dd^3\vv{x} \inv{2\sqrt{\omega_{\vv{p}} \omega_{\vv{p}'}}} \Big[ \pexp{\mp ipx} \big\{\pm i \omega_{\vv{p}'} \pexp{\pm ip'x} \big\} \\
            &\qquad\qquad\qquad\qquad\qquad + \big\{ \pm i \omega_{\vv{p}} \pexp{\mp ipx} \big\} \pexp{\pm ip'x} \Big] \\
        &= \pm i^2 \int \dd^3\vv{x} \inv{2} \left[ \sqf{\omega_{\vv{p}'}}{\omega_{\vv{p}}} + \sqf{\omega_{\vv{p}}}{\omega_{\vv{p}'}} \right] \pexp{\mp ipx} \pexp{\pm ip'x} \\
        &= \mp \int \dd^3\vv{x} \inv{2} \left[ \sqf{\omega_{\vv{p}'}}{\omega_{\vv{p}}} + \sqf{\omega_{\vv{p}}}{\omega_{\vv{p}'}} \right] \pexp{\mp i[p - p']x} \\
        &= \mp \inv{2} \left[ \sqf{\omega_{\vv{p}'}}{\omega_{\vv{p}}} + \sqf{\omega_{\vv{p}}}{\omega_{\vv{p}'}} \right] (2\pi)^3 \delta^3(\vv{p} - \vv{p}') \pexp{\mp i[\omega_{\vv{p}} - \omega_{\vv{p}'}]t} \\
        &= \mp \inv{2} \left[ \sqf{\omega_{\vv{p}}}{\omega_{\vv{p}}} + \sqf{\omega_{\vv{p}}}{\omega_{\vv{p}}} \right] (2\pi)^3 \delta^3(\vv{p} - \vv{p}') \pexp{\mp i[\omega_{\vv{p}} - \omega_{\vv{p}}]t} \\
        &= \mp \inv{2} \left[ 1 + 1 \right] (2\pi)^3 \delta^3(\vv{p} - \vv{p}') \pexp{0} \\
        &= \mp (2\pi)^3 \delta^3(\vv{p} - \vv{p}') \: ,
\end{split}
\end{align}
altså
\begin{subequations}
\begin{align}
    (u_{\vv{p}},u_{\vv{p}}) &= (2\pi)^3 \delta^3(\vv{p} - \vv{p}') \quad \text{og} \\
    (u_{\vv{p}}^*,u_{\vv{p}}^*) &= - (2\pi)^3 \delta^3(\vv{p} - \vv{p}') \: ,
\end{align}
\end{subequations}
idet at deltafunktioner fremkommer ved $\int \dd k \pexp{iky} = 2\pi\delta(y)$, at $\delta(y-y') = \delta(y'-y)$ og at $f(y') \delta(y - y') = f(y) \delta(y - y')$.

Dernæst beregnes $(u_{\vv{p}}^*,u_{\vv{p}'})$ og $(u_{\vv{p}},u_{\vv{p}'}^*)$:
\begin{align}
\begin{split}
    (u_{\vv{p}}^\pm,u_{\vv{p}}^\mp) &= i \int \dd^3\vv{x} \left[ u_{\vv{p}}^\mp \pdif{u_{\vv{p}'}^\mp}{t} - \pdif{u_{\vv{p}}^\mp}{t} u_{\vv{p}'}^\mp \right] \\
        &= i \int \dd^3\vv{x} \inv{2\sqrt{\omega_{\vv{p}} \omega_{\vv{p}'}}} \Big[ \pexp{\mp ipx} \big\{\mp i \omega_{\vv{p}'} \pexp{\mp ip'x} \big\} \\
            &\qquad\qquad\qquad\qquad\qquad - \big\{ \mp i \omega_{\vv{p}} \pexp{\mp ipx} \big\} \pexp{\mp ip'x} \Big] \\
        &= \mp i^2 \int \dd^3\vv{x} \inv{2\sqrt{\omega_{\vv{p}} \omega_{\vv{p}'}}} \Big[ \pexp{\mp ipx} \big\{\omega_{\vv{p}'} \pexp{\mp ip'x} \big\} \\
            &\qquad\qquad\qquad\qquad\qquad - \big\{ \omega_{\vv{p}} \pexp{\mp ipx} \big\} \pexp{\mp ip'x} \Big] \\
        &= \mp \int \dd^3\vv{x} \inv{2} \left[ \sqf{\omega_{\vv{p}'}}{\omega_{\vv{p}}} - \sqf{\omega_{\vv{p}}}{\omega_{\vv{p}'}} \right] \pexp{\mp i[p + p']x} \\
        &= \mp \inv{2} \left[ \sqf{\omega_{\vv{p}'}}{\omega_{\vv{p}}} - \sqf{\omega_{\vv{p}}}{\omega_{\vv{p}'}} \right] (2\pi)^3 \delta^3(\vv{p} + \vv{p}') \pexp{\mp i[\omega_{\vv{p}} + \omega_{\vv{p}'}]t} \\
        &= \mp \inv{2} \left[ \sqf{\omega_{-\vv{p}}}{\omega_{\vv{p}}} - \sqf{\omega_{\vv{p}}}{\omega_{-\vv{p}}} \right] (2\pi)^3 \delta^3(\vv{p} + \vv{p}') \pexp{\mp i[\omega_{\vv{p}} + \omega_{-\vv{p}}]t} \\
        &= \mp \inv{2} \left[ \sqf{\omega_{\vv{p}}}{\omega_{\vv{p}}} - \sqf{\omega_{\vv{p}}}{\omega_{\vv{p}}} \right] (2\pi)^3 \delta^3(\vv{p} + \vv{p}') \pexp{\mp i[\omega_{\vv{p}} + \omega_{\vv{p}}]t} \\
        &= 0 \: ,
\end{split}
\end{align}
idet
\begin{align}
    \omega_{\vv{p}} &= \sqrt{\vv{p}^2 + m^2}
        \quad \Rightarrow \quad
    \omega_{-\vv{p}} = \sqrt{\left(-\vv{p}\right)^2 + m^2} = \sqrt{\vv{p}^2 + m^2} = \omega_{\vv{p}} \: ,
\end{align}
altså får vi at
\begin{align}
    (u_{\vv{p}}^*,u_{\vv{p}}) &= (u_{\vv{p}},u_{\vv{p}}^*) = 0 \: .
\end{align}

Dermed har vi vist de fire indre produkter i \cref{eq:Opg2_Q7_KGInnerProductsWithU}.


%%%%%%%%%%%%%%%%%%%%%%%%%

\paragraph[8) Relationer for kreations- og annihilationsoperatorerne med Klein-Gordon-indreproduktet]{\textbf{8)}}

Definerer vi, som i besvarelsen til \textbf{7)}, $u_{\vv{p}} = u_{\vv{p}}^-$ og $u_{\vv{p}}^* = u_{\vv{p}}^+$ for short-hand notation og husker, at $u_{\vv{p}}^\pm = \pexp{\pm ipx}/\sqrt{2\omega_{\vv{p}}}$, hvorfor
\begin{align}
\begin{split}
    \phi_{ij} &= \int \frac{\dd^3\vv{p}}{(2\pi)^3} \invsqrt{2\omega_{\vv{p}}} \left(a_{\vv{p}ij} \bexp{-ipx} + a_{\vv{p}ij}\dagger \bexp{ipx} \right) \\
        &= \int \frac{\dd^3\vv{p}}{(2\pi)^3} \left(a_{\vv{p}ij} u_{\vv{p}}^- + a_{\vv{p}ij}\dagger u_{\vv{p}}^+ \right) \: ,
\end{split}
\end{align}
så får vi, at
\begin{align} \label{eq:Opg2_A8_CalculationOfStepOperators}
\begin{split}
    \big(u_{\vv{p}}^\pm,\phi_{ij}(\vv{p}')\big)
        &= i \int \dd^3\vv{x} \left[ u_{\vv{p}}^\mp \pdif{\phi_{ij}(\vv{p}')}{t} - \pdif{u_{\vv{p}}^\mp}{t} \phi_{ij}(\vv{p}') \right] \\
        &= i \int \dd^3\vv{x} \bigg[ u_{\vv{p}}^\mp \pdif{}{t}\left( \int \frac{\dd^3\vv{p}'}{(2\pi)^3} \left\{a_{\vv{p}'ij} u_{\vv{p}'}^- + a_{\vv{p}'ij}\dagger u_{\vv{p}'}^+ \right\} \right) \\
            &\qquad\qquad\quad - \pdif{u_{\vv{p}}^\mp}{t} \left( \int \frac{\dd^3\vv{p}'}{(2\pi)^3} \left\{a_{\vv{p}'ij} u_{\vv{p}'}^- + a_{\vv{p}'ij}\dagger u_{\vv{p}'}^+ \right\} \right) \bigg] \\
        &= i \int \frac{\dd^3\vv{x} \, \dd^3\vv{p}'}{(2\pi)^3} \Bigg[ u_{\vv{p}}^\mp \pdif{}{t}\left( a_{\vv{p}'ij} u_{\vv{p}'}^- + a_{\vv{p}'ij}\dagger u_{\vv{p}'}^+ \right) \\
            &\qquad\qquad\qquad\quad - \pdif{u_{\vv{p}}^\mp}{t} \left( a_{\vv{p}'ij} u_{\vv{p}'}^- + a_{\vv{p}'ij}\dagger u_{\vv{p}'}^+ \right) \Bigg] \\
        &= i \int \frac{\dd^3\vv{x} \, \dd^3\vv{p}'}{(2\pi)^3} \Bigg[ u_{\vv{p}}^\mp \left( a_{\vv{p}'ij} \pdif{u_{\vv{p}'}^-}{t} + a_{\vv{p}'ij}\dagger \pdif{u_{\vv{p}'}^+}{t} \right) \\
            &\qquad\qquad\qquad\quad - \pdif{u_{\vv{p}}^\mp}{t} \left( a_{\vv{p}'ij} u_{\vv{p}'}^- - a_{\vv{p}'ij}\dagger u_{\vv{p}'}^+ \right) \Bigg] \\
        &= i \int \frac{\dd^3\vv{x} \, \dd^3\vv{p}'}{(2\pi)^3} \Bigg[ a_{\vv{p}'ij} \left( u_{\vv{p}}^\mp \pdif{u_{\vv{p}'}^-}{t} - \pdif{u_{\vv{p}}^\mp}{t} u_{\vv{p}'}^- \right) \\
            &\qquad\qquad\qquad\quad + a_{\vv{p}'ij}\dagger \left( u_{\vv{p}}^\mp \pdif{u_{\vv{p}'}^+}{t} - \pdif{u_{\vv{p}}^\mp}{t} u_{\vv{p}'}^+ \right) \Bigg] \\
        &= \int \frac{\dd^3\vv{p}'}{(2\pi)^3} \bigg[ a_{\vv{p}'ij} ( u_{\vv{p}}^\pm , u_{\vv{p}'}^- ) + a_{\vv{p}'ij}\dagger ( u_{\vv{p}}^\pm , u_{\vv{p}'}^+ ) \bigg] \: ,
\end{split}
\end{align}
hvor Leibniz integrationsregel\footnote{
    Leibniz integrationsregl er \cite{wiki:leibnizIntegrationRule}
    \begin{align*}
        \dif{}{x}\left( \int_{a(x)}^{b(x)} \dd y \, f(x,y) \right) &= f\big(x, b(x)\big) \dif{b(x)}{x} - f\big(x, a(x)\big) \dif{a(x)}{x} + \int_{a(x)}^{b(x)} \dd y \, \pdif{f(x,y)}{x} \: .
    \end{align*}
    Når $a(x)$ og $b(x)$ er konstanter i stedet for funktioner ($a$, $b$), så bliver Leibniz integrationsregl
    \begin{align*}
        \dif{}{x}\left( \int_a^b \dd y \, f(x,y) \right) &= \int_a^b \dd y \, \pdif{f(x,y)}{x} \: .
    \end{align*}
}
er blevet benyttet for at bytte integralet og den partielle differentialoperator, og der ingen ekstra faktor kommer, da grænserne er konstanter.

Fra \cref{eq:Opg2_A8_CalculationOfStepOperators} får vi altså, at
\begin{subequations}
\begin{align}
    \begin{split}
        (u_{\vv{p}},\phi_{ij}) &= \int \frac{\dd^3\vv{p}'}{(2\pi)^3} \Big[ a_{\vv{p}'ij} ( u_{\vv{p}} , u_{\vv{p}'} ) + a_{\vv{p}'ij}\dagger ( u_{\vv{p}} , u_{\vv{p}'}^* ) \Big] \\
            &= \int \frac{\dd^3\vv{p}'}{(2\pi)^3} \Big[ a_{\vv{p}'ij} (2\pi)^3\delta^3(\vv{p} - \vv{p}') + 0 \Big] \\
            &= a_{\vv{p}ij}\dagger \: ,
    \end{split}
    \\
    \begin{split}
        (u_{\vv{p}}^*,\phi_{ij}) &= \int \frac{\dd^3\vv{p}'}{(2\pi)^3} \Big[ a_{\vv{p}'ij} ( u_{\vv{p}}^* , u_{\vv{p}'} ) + a_{\vv{p}'ij}\dagger ( u_{\vv{p}}^* , u_{\vv{p}'}^* ) \Big] \\
            &= \int \frac{\dd^3\vv{p}'}{(2\pi)^3} \Big[ 0 + a_{\vv{p}'ij}\dagger  \big\{ -(2\pi)^3\delta^3(\vv{p} - \vv{p}') \big\} \Big] \\
            &= -a_{\vv{p}ij}\dagger \: ,
    \end{split}
\end{align}
\end{subequations}
hvilke er \cref{eq:Opg2_Q8_StepOperatorsUsingKGInnerProduct}, som vi skulle vise.


%%%%%%%%%%%%%%%%%%%%%%%%%

\paragraph[9) Bosonisk kommutationsrelation for kreations- og annihilations-operatorerne]{\textbf{9)}}




%%%%%%%%%%%%%%%%%%%%%%%%%

\paragraph[10) Hamiltonfunktion med kreations- og annihilationsoperatorer]{\textbf{10)}}




%%%%%%%%%%%%%%%%%%%%%%%%%

\paragraph[11) Feltoperator overholder Heisenbergs bevægelsesligning]{\textbf{11)}}




%%%%%%%%%%%%%%%%%%%%%%%%%

\paragraph[12) Lagrangedensitet invariant under $O(N)^2$ transformation]{\textbf{12)}}




%%%%%%%%%%%%%%%%%%%%%%%%%

\paragraph[13) Vekselvirkning med og uden $O(N)^2$ symmetri]{\textbf{13)}}




%%%%%%%%%%%%%%%%%%%%%%%%%

\paragraph[14) Generatoren af $O(N)$]{\textbf{14)}}




%%%%%%%%%%%%%%%%%%%%%%%%%

\paragraph[15) Noetherstrøm associeret med $O(N)^2$ symmetri]{\textbf{15)}}




%%%%%%%%%%%%%%%%%%%%%%%%%

\paragraph[16) Ændringer hvis $\phi_{ij}$ komplekst i stedet for reelt]{\textbf{16)}}




%%%%%%%%%%%%%%%%%%%%%%%%%%%%%%%%%%%%%%%%%%%%%%%%%%%%%%%%%%%%%%%%%%%%%%%%%%%%%%%%%%%%%

\end{document}